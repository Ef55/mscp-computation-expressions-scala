\label{sec:caveats}

The tricks introduced in the previous section have some serious caveats,
that we will now discuss.

The most glaring one is the use of implicit conversions. 
These are known to be a kind of pandora's box of confusing behaviors; here however, the issue goes a bit deeper.
The issue is that those conversions are ``contextual'',
they only make sense in a particular case: the \lstinline{Boolean} one only makes sense when used in a control-flow structure;
the tree to variable one should only be used to assign to a variable;
the \lstinline{Unit} one is only meaningful when applied to an assignation to a user's AST variable.

\lstinputlisting[language=scala]{code/invalid_conversions.scala}

It is possible to mitigate this by implementing the conversions such that they throw an exception (at runtime) when called.
Since they are erased anyway when used correctly, this will only throw when used incorrectly. 

Even though practical and easy to implement, 
this idea is terrible for the user, as something which is statically known results in a runtime error, which is dirty.

It is possible to report such misuse during AST transformation, by detecting and reporting unwanted conversions, and then stopping compilation.
This however is error-prone and requires to correctly predict every corner-case.

This motivated the abandon of features which relied too heavily on implicit conversions; 
more precisely, if-then-else, while-loops, variable assignations and user-level variables inference are not supported by the library.

Instead, a special operator is used for user-level variables definition and assignation.
To be nice to use, the operator for user-level variables had to be a prefix operator; since scala only allows a handful of prefix operators (\scalasnippet{+, -, !, ~}),
\scalasnippet{!} was chosen since it reminds of \Fsharp{} computation expressions (\cref{sec:ce}).
The symbol \scalasnippet{=!} was chosen for assignation to be coherent with \scalasnippet{!}, even though \scalasnippet{:=} could have been a better choice (it is both already known and has no risk of being confused with \scalasnippet{!=}).
The reflection API allows to test for variable mutability,
as such assignation (using \lstinline{=!}) on non-mutable variables can be detected and reported.

Replacements can be imported using a mixin for the control-flow structures, which are simply implemented as functions with judicious names and currying of arguments.

\begin{figure}[h]
\lstinputlisting[language=scala]{code/dsl_over_conv.scala}
\caption{Snippet of code showing the notations used.}
\end{figure}