\subsection{A motivating example}

Simple planes \cite{simplePlanes} is a computer game where you have to create your own planes, which you can then fly.

More interestingly, it is possible to write a program which will control the plane using a dedicated programming language, 
funky trees \cite{funkyTrees}.
This language is however expressed in XML,
doesn't directly support usual control-flow structures
and isn't type safe.

This makes this language both difficult to explain to people without programming experience, as well as quite frustrating to people with such experience.
This is quite a shame,
as the highly visual and concrete aspect of controlling a plane could make a great pedagogical tool to teach computer science.

\subsection{Designing a better language}
One solution to this issue would be to design a 
``higher-level'' programming language for simple planes,
treating XML as its ``low-level'' assembly.

However, designing such a language and implementing a full compiler for it
would quickly become time consuming, and would most likely result in pretty minimal (feature-wise) language.

Another quicker solution would be to use an pre-existing language,
and embed our language in it directly, resulting in a domain-specific language (DSL) \cite{langAndTools}.

\Scala{}, being a feature-rich language, makes a pretty good candidate for such usages;
e.g. it allows custom operators definitions and has excellent support for algebraic data types.

Thus, an attempt at writing our DSL could yield something along these lines:
\lstinputlisting[language=scala]{code/plane_example.scala}

We can however already notice some downsides, among others:
\begin{itemize}
    \item The variables are named twice and must be declared before all of the actual code;
    \item An operator must be used for sequencing (\scalasnippet{andThen}).
\end{itemize}

Overall, the user ends up manipulating the resulting AST a bit too directly, despite the DSL making it a bit more natural.
This is especially problematic, as the end-user (the one using the DSL) might be learning programming, in which case 
manipulating a program (for the plane) using a program (the one in \Scala{}) might be confusing.
Even for more seasoned programmers, this can be confusing if they are not used to computer languages representation.

\subsection{Metaprogramming}
One possible way to alleviate the mentioned issues is to use metaprogramming.
In particular, \Scala{} has some young yet powerful metaprogramming facilities \cite{metaprogramming}.

\begin{tcolorbox}
\paragraph{Aside} 
From now on, we will be talking about multiple kinds of AST at the same time:
the AST of the \Scala{} program, which will be manipulated through metaprogramming,
and the AST which is the results of this program, built using the DSL and thus defined by the user.
We will refer to the former as ``\Scala{}'s AST'' and to the later as ``user's AST''.
\end{tcolorbox}

In fact, these facilities allow to manipulate the \Scala{}'s program AST directly.
This would allow to define some function on ASTs such that
\[ \codefrag{ t_1; t_2} \longmapsto \textsc{sequence}(t_1, t_2) \ \]
Where $\textsc{sequence}$ is some function (or constructor) producing a user AST.

\subsection{Goals}
The goal of this project, as well as report, is thus to explore how metaprogramming can be used for AST generation.
More precisely, starting from the less-funky-tree compiler \cite{less-funky-trees},
it aims at identifying elements which could be generalized and abstracted,
as to create a metaprogramming-powered DSL library.

These ideas being also kind of reminiscent of \Fsharp{}'s computation expressions \cite{ceZoo},
this project will also aim at seeing potential common abstractions as well as differences between the two.
