\newcommand{\treetype}[0]{\lstinline{Tree[_]}}

The core idea to solve the issue at hand is to use Scala metaprogramming,
more precisely its reflection capabilities, which allow to directly manipulate the program's AST.

The problem can then be split into the following sub-problems:
\begin{enumerate}
    \item Allow the DSL designer to specify how to transform a Scala construct into its counterpart; \label{sprob:spec}
    \item Make the DSL type-check; \label{sprob:typecheck}
    \item Given a Scala AST, traverse it and identify which instances of the different constructs (variable definition, expression sequencing, \ldots{}) have to be transformed into their user's AST counterpart. \label{sprob:transform}
\end{enumerate}
\newcommand{\refprob}[1]{{(\ref{sprob:#1})}}


\refprob{spec} can easily be solved using a trait,
that the DSL developer will have to implement:
\vspace{4em}
\lstinputlisting[language=scala]{code/ast_builder_early.scala}
The type parameter \treetype{} is the type of the produced AST (i.e. the user's AST), and \scalasnippet{Variable[_]} is used to represent variables inside of the user's AST. 
Its type parameter allows the user to have type-safety inside the DSL, e.g.
it allows to add a type-safe \scalasnippet{+} operator to \rob{}'s DSL:
\lstinputlisting[language=scala]{code/tree_int_ext.scala}


With the previous solution,
the transform part of \refprob{transform} can be done by simply replacing the Scala construct by a call to the correct method. 
We can also decide to apply the transformation if and only if its operand have the user's AST type (i.e \scalasnippet{AST[_]} in \rob{}'s case). A simple example would be:

\translation{code/type_guide}

Here, \scalasnippet{i} is not transformed because it does not have type \scalasnippet{Variable[_]}, whereas
\scalasnippet{t} has.

Finally, a simple solution to \refprob{typecheck} is to introduce implicit conversion when needed:
\translation{code/type_check_conv}
This code sample won't compile for multiple reasons:
\begin{itemize}
    \item The condition of a control-flow structure must be a \scalasnippet{Boolean};
    \item The variable initialization as well as both assignations have type \scalasnippet{Tree[Int]} as their right-hand-side, whereas the left-hand-side is a \scalasnippet{Variable[Int]}.
\end{itemize}
Also, the fact that both assignations have type \scalasnippet{Unit}, whereas they will have type \scalasnippet{Tree[Unit]} after translation, can disallow some code we would like to allow.
However, if implicit conversions are added, more precisely
\begin{itemize}
    \item \scalasnippet{Conversion[Tree[Boolean], Boolean]},
    \item \scalasnippet{Conversion[Tree[T], Variable[T]]} and
    \item \scalasnippet{Conversion[Unit, Tree[Unit]]}
\end{itemize}
this codes is then accepted by the compiler.
These implicit conversions do not ``do'' anything:
they simply indicate to the compiler that such programs should compile, and are erased during transformation,
as the resulting user's AST does not require a conversions e.g. from \scalasnippet{Tree[Boolean]} to \scalasnippet{Boolean]} inside the condition of an if-then-else.
